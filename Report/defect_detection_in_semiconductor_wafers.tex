\documentclass[conference]{IEEEtran}
\IEEEoverridecommandlockouts
% The preceding line is only needed to identify funding in the first footnote. If that is unneeded, please comment it out.
\usepackage{cite}
\usepackage{amsmath,amssymb,amsfonts}
\usepackage{algorithmic}
\usepackage{graphicx}
\usepackage{textcomp}
\usepackage{xcolor}
\def\BibTeX{{\rm B\kern-.05em{\sc i\kern-.025em b}\kern-.08em
    T\kern-.1667em\lower.7ex\hbox{E}\kern-.125emX}}
\begin{document}

\title{
    Defect Detection in Semiconductor Wafers Using Image Classification
    \thanks{Computational resources were provided by the WVU Research Computing Thorny Flat HPC cluster, partly funded by NSF OAC-1726534.}
}

% Authors 
\author{
    \IEEEauthorblockN{Ian S. Jackson}
    \IEEEauthorblockA{\textit{Lane Department of Computer Science and Electrical Engineering} \\
    \textit{West Virginia University}\\
            Morgantown, United States \\
            isj0001@mix.wvu.edu}
}

\maketitle
\thispagestyle{plain}
\pagestyle{plain}

\begin{abstract}
    ABSTRACT
\end{abstract}

%-- Introduction --%
\section{Introduction}
Defect detection is a critical process in the semiconductor manufacturing industry. 
Traditional inspection methods often rely on manual analysis or rule-based systems, which can be time-consuming and prone to human error.
Additionally, the increasing complexity and miniaturization of semiconductor devices necessitate highly precise defect detection methods to ensure manufacturing yield and device reliability.
Semiconductor wafer defects, such as cracks, scratches, and contamination, can significantly impact the performance and lead to costly failures. 
Recent advancements in deep learning and computer vision have enabled automated defect detection using image classification models, offering improved accuracy and efficiency \cite{b1}.

In this work, I propose to explore and analyze deep learning-based and pattern recognition-based approaches for multi-class defect classification in semiconductor wafers. 
Specifically, the project will investigate convolutional neural networks (CNNs), support vector machines (SVMs), and k-nearest neighbors (KNNs) for their effectiveness in classifying defects from high-resolution wafer images. 
% CNNs, known for their outstanding performance in image recognition tasks \cite{b2}, will be a primary focus. 
Additionally, a fusion model that integrates the strengths of these methods will be developed to further enhance classification performance.

%-- Background --%
\section{Background}


%-- Related Work --%
\section{Related Work} 


%-- Approach --%
\section{Approach} 
This section outlines the methodology for defect detection in semiconductor wafers using image classification techniques.
The approach consists of several key steps: dataset selection and preprocessing, model development, training, and evaluation.

%- Dataset Description -%
\subsection{Dataset Description}


%-- Results --%
\section{Results} 


%-- Discussion --%
\section{Discussion}


%-- Conclusion --%
\section{Conclusion}


\begin{thebibliography}{00}
\bibitem{b1} Y. LeCun, Y. Bengio, and G. Hinton, “Deep learning,” Nature, vol. 521, no. 7553, pp. 436–444, 2015.
% \bibitem{b2} S. Khan, M. Hayat, M. Bennamoun, F. A. Sohel, and R. Togneri, “A guide to convolutional neural networks for computer vision,” ACM Computing Surveys (CSUR), vol. 51, no. 5, pp. 1–58, 2020.

\end{thebibliography}
\end{document}
