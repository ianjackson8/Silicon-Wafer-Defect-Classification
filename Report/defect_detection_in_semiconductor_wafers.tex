\documentclass[conference]{IEEEtran}
\IEEEoverridecommandlockouts
% The preceding line is only needed to identify funding in the first footnote. If that is unneeded, please comment it out.
\usepackage{cite}
\usepackage{amsmath,amssymb,amsfonts}
\usepackage{algorithmic}
\usepackage{graphicx}
\usepackage{textcomp}
\usepackage{xcolor}
\def\BibTeX{{\rm B\kern-.05em{\sc i\kern-.025em b}\kern-.08em
    T\kern-.1667em\lower.7ex\hbox{E}\kern-.125emX}}
\begin{document}

\title{
    Defect Detection in Semiconductor Wafers Using Image Classification
}

% Authors 
\author{
    \IEEEauthorblockN{Ian S. Jackson}
    \IEEEauthorblockA{\textit{Lane Department of Computer Science and Electrical Engineering} \\
    \textit{West Virginia University}\\
            Morgantown, United States \\
            isj0001@mix.wvu.edu}
}

\maketitle

\begin{abstract}
    ABSTRACT
\end{abstract}

\section{Introduction}
Defect detection is a critical process in the semiconductor manufacturing industry. 
Traditional inspection methods often rely on manual analysis or rule-based systems, which can be time-consuming 
and prone to human error.
Additionally, the increasing complexity and miniaturization of semiconductor devices necessitate highly precise 
defect detection methods to ensure manufacturing yield and device reliability.
Semiconductor wafer defects, such as cracks, scratches, and contamination, can significantly impact the performance
and lead to costly failures. 
Recent advancements in deep learning and computer vision have enabled automated defect detection using image classification
models, offering improved accuracy and efficiency \cite{b1}.

In this work, I propose a deep learning-based approach for defect detection in semiconductor wafers using image classification.
Convolutional neural networks (CNNs) have demonstrated exceptional performance in image recognition tasks and are well
suited for identifying defect patterns in high-resolution wafer images \cite{b2}. 
In addition to CNNs, other paten recognition methods, such as support vector machines (SVM) and k-nearest neighbors (KNN),
can be implemented and tested for defect classification. 

\section{Background}


\section{Related Work} 


\section{Approach} 


\section{Results} 


\section{Conclusion}


\begin{thebibliography}{00}
\bibitem{b1} Y. LeCun, Y. Bengio, and G. Hinton, “Deep learning,” Nature, vol. 521, no. 7553, pp. 436–444, 2015.
\bibitem{b2} S. Khan, M. Hayat, M. Bennamoun, F. A. Sohel, and R. Togneri, “A guide to convolutional neural networks for computer vision,” ACM Computing Surveys (CSUR), vol. 51, no. 5, pp. 1–58, 2020.

\end{thebibliography}
\end{document}
